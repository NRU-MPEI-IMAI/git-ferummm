\documentclass{article}
\usepackage[utf8]{inputenc}
\usepackage[russian]{babel}
\usepackage[letterpaper,top=2cm,bottom=2cm,left=2cm,right=1cm,marginparwidth=1.25cm]{geometry}
\usepackage[pdf]{graphviz}
\usepackage{xpatch}
\usepackage{morewrites}
\usepackage[utf8]{inputenc}
\usepackage[T2A]{fontenc}
 
\title{ТМВ}
\author{Полина Андреева}
\date{April 2022}

\begin{document}


\maketitle

\makeatletter
\newcommand*{\addFileDependency}[1]{% argument=file name and extension
  \typeout{(#1)}
  \@addtofilelist{#1}
  \IfFileExists{#1}{}{\typeout{No file #1.}}
}
\makeatother
\xpretocmd{\digraph}{\addFileDependency{#2.dot}}{}{}

\section{Задание №1. Построить конечный автомат, распознающий язык}

\begin{center}
\quad $1.1) \quad L = \{ w \in \{ a, b, c\}^* | \quad |w|_c = 1 \}$

\digraph [scale=1]{d1}{
    rankdir=LR;
    0 [label = < >, shape = none,height=.0,width=.0]
    1 [shape = "circle"]
    2 [shape = "doublecircle"]
    0->1
    1->1 [label="a,b" ]
    1->2 [label="c" ]
    2->2 [label="a,b"]
}

$1.2) \quad L = \{ w \in \{ a, b\}^* | \quad |w|_a \leq 2, |w|_b \geq 2 \}$ 
\end{center}
Найдем прямое произведение двух автоматов, распознающих языки:

\begin{center}
$L1 = \{ w \in \{ a, b\}^* | \quad |w|_a \leq 2\}$ \\

\digraph [scale=1]{d2}{
    rankdir=LR;
    0 [label = < >, shape = none,height=.0,width=.0]
    q1,q2,q3 [shape = "doublecircle"]
    0->q1
    q1->q1 [label="b"]
    q1->q2 [label="a"]
    q2->q2 [label="b"]
    q2->q3 [label="a"]
    q3->q3 [label="b"]
}


$L2 = \{ w \in \{ a, b\}^* | \quad |w|_b \geq 2 \}$\\

\digraph [scale=1]{d3}{
    rankdir=LR;
    0 [label = < >, shape = none,height=.0,width=.0]
    s1,s2 [shape = "circle"]
    s3 [shape = "doublecircle"]
    0->s1
    s1->s1 [label="a"]
    s1->s2 [label="b"]
    s2->s2 [label="a"]
    s2->s3 [label="b"]
    s3->s3 [label="a,b"]
}

 \begin{tabular}{|c|c|c|}
            \hline
            Состояния & a & b \\
            \hline
            q1s1 & q2s1 & q1s2\\
            q1s2 & q2s2 & q1s3\\
            \textbf{q1s3} & q2s3 & q1s3\\
            q2s1 & q3s1 & q2s2\\
            q2s2 & q3s2 & q2s3\\
            \textbf{q2s3} & q3s3 & q2s3\\
            q3s1 &  & q3s2\\
            q3s2 &  & q3s3\\
            \textbf{q3s3} &  & q3s3\\
            \hline
\end{tabular}

\digraph [scale=0.7]{d4}{
    rankdir=LR;
    0 [label = < >, shape = none,height=.0,width=.0]
    q1s1,q1s2,q2s1,q2s2,q3s1,q3s2 [shape = "circle"]
    q1s3,q2s3,q3s3 [shape = "doublecircle"]
    0->q1s1
    q1s1->q2s1[label="a"]
    q1s2->q2s2[label="a"]
    q1s3->q2s3[label="a"]
    q2s1->q3s1[label="a"]
    q2s2->q3s2[label="a"]
    q2s3->q3s3[label="a"]
    
    q1s1->q1s2[label="b"]
    q1s2->q1s3[label="b"]
    q1s3->q1s3[label="b"]
    q2s1->q2s2[label="b"]
    q2s2->q2s3[label="b"]
    q2s3->q2s3[label="b"]
    
    q3s1->q3s2[label="b"]
    q3s2->q3s3[label="b"]
    q3s3->q3s3[label="b"]

}
$1.3) \quad L = \{ w \in \{ a, b\}^* | \quad |w|_a \neq |w|_b \}$ 
\end{center}

Данный язык не является регулярным, так как является дополнением к нерегулярному языку $\overline L$ (в силу леммы о разрастании и замкнутости относительно операции дополнения):

\begin{center}
$\overline L = \{ w \in \{ a, b\}^* | \quad |w|_a = |w|_b \}$  
\end{center}

Так как язык $L$ нерегулярный автомат построить нельзя.

\begin{center}
$1.4) \quad L = \{ w \in \{ a, b\}^* | \quad ww = www \}$
\end{center} 

Данный язык включает в себя только пустое слово.

\begin{center}
\digraph [scale=1]{d5}{
    rankdir=LR;
    0 [label = < >, shape = none,height=.0,width=.0]
    1 [shape = "doublecircle"]
    0->1
}

\end{center} 

Также этот автомат можно изобразить следующим образом, включая в него тупиковое состояние:
\begin{center}
\digraph [scale=1]{d6}{
    rankdir=LR;
    0 [label = < >, shape = none,height=.0,width=.0]
    1 [shape = "doublecircle"]
    2 [shape = "circle"]
    0->1
    1->2 [label="a,b"]
    2->2 [label="a,b"]
    }
\end{center} 

Последние два автомата эквивалентны.


\section{Задание №2. Построить конечный автомат, используя прямое произведение}

\begin{center}
$2.1) \quad L_1 = \{w \in \{ a,b \}^*  | \quad |w|_a \geq 2 \wedge |w|_b \geq 2 \}$
\end{center}



\begin{center}
$L_1_1 = \{w \in \{ a,b \}^* \; | \quad |w|_a \geq 2 \}$


\digraph[scale = 0.7] {d7} {
    rankdir = LR;
    0 [label = < >, shape = none,height=.0,width=.0]
    q1, q2 [shape = circle]
    q3 [shape = doublecircle]
    0->q1
    q1->q2 [label = "a"];
    q1->q1 [label = "b"];
    q2->q3 [label = "a"];
    q2->q2 [label = "b"];
    q3->q3 [label = "a,b"];
}
\end{center}


\begin{center}
$L_1_2 = \{w \in \{ a,b \}^* \; | \quad |w|_b \geq 2 \}$


\digraph[scale = 0.7] {d8} {
    rankdir = LR
    0 [label = < >, shape = none,height=.0,width=.0]
    s1, s2 [shape = circle]
    s3 [shape = doublecircle]
    0->s1
    s1->s2 [label = "b"];
    s1->s1 [label = "a"];
    s2->s3 [label = "b"];
    s2->s2 [label = "a"];
    s3->s3 [label = "a,b"];
}


 \begin{tabular}{|c|c|c|}
            \hline
            Состояния & a & b \\
            \hline
            q1s1 & q2s1 & q1s2\\
            q1s2 & q2s2 & q1s3\\
            q1s3 & q2s3 & q1s3\\
            q2s1 & q3s1 & q2s2\\
            q2s2 & q3s2 & q2s3\\
            q2s3 & q3s3 & q2s3\\
            q3s1 & q3s1 & q3s2\\
            q3s2 & q3s2 & q3s3\\
            \textbf{q3s3} & q3s3 & q3s3\\
            \hline
\end{tabular}
\newline
\end{center}

Прямое произведение автоматов:

\begin{center}
\digraph[scale=0.7] {d9} {
    rankdir = LR
    0 [label = < >, shape = none,height=.0,width=.0]
    q1s1,q1s2,q1s3,q2s1,q2s2,q2s3,q3s1,q3s2 [shape = circle]
    q3s3 [shape = doublecircle]
    0->q1s1
    q1s1->q2s1[label="a"]
    q1s2->q2s2[label="a"]
    q1s3->q2s3[label="a"]
    q2s1->q3s1[label="a"]
    q2s2->q3s2[label="a"]
    q2s3->q3s3[label="a"]
    q3s1->q3s1[label="a"]
    q3s2->q3s2[label="a"]
    
    q1s1->q1s2[label="b"]
    q1s2->q1s3[label="b"]
    q1s3->q1s3[label="b"]
    q2s1->q2s2[label="b"]
    q2s2->q2s3[label="b"]
    q2s3->q2s3[label="b"]
    q3s1->q3s2[label="b"]
    q3s2->q3s3[label="b"]
    q3s3->q3s3[label="a,b"]
}



$2.2) \quad L_2 = \{w \in \{ a,b \}^*  | \quad |w| \geq 3 \wedge |w|$ нечетное$\}$ \newline


$\quad L_2_1 = \{w \in \{ a,b \}^*  | \quad |w| \geq 3\}$ \newline
\digraph[scale=0.7]{d10}{
    rankdir=LR
    0 [label = < >, shape = none,height=.0,width=.0]
    q4 [shape = doublecircle]
    q1,q2,q3 [shape = circle]
     
    0 -> q1
    q1 -> q2 [label = "a, b"]
    q2 -> q3 [label = "a, b"]
    q3 -> q4 [label = "a, b"]
    q4 -> q4 [label = "a, b"]
}
            
            
$\quad L_2_2 = \{w \in \{ a,b \}^*  | \quad |w|$ нечетное$\}$   \newline         
\digraph[scale=0.7]{d11}{
    0 [label = < >, shape = none,height=.0,width=.0]
    s2 [shape = doublecircle]
    s1 [shape = circle]
    rankdir=LR; 
    0 -> s1;
    s1 -> s2 [label = "a, b"];
    s2 -> s1 [label = "a, b"];
}    

\begin{tabular}{|c|c|c|}
            \hline
            Состояния & a & b \\
            \hline
            q1s1 & q2s2 & q2s2\\
            q1s2 & q2s1 & q2s1\\
            q2s1 & q3s2 & q3s2\\
            q2s2 & q3s1 & q3s1\\
            q3s1 & q4s2 & q4s2\\
            q3s2 & q4s1 & q4s1\\
            q4s1 & q4s2 & q4s2\\
            \textbf{q4s2} & q4s1 & q4s1\\
            \hline
\end{tabular}
\newline
\end{center}

Прямое произведение автоматов:

\begin{center}
\digraph[scale=0.7] {d12} {
    rankdir = LR
    0 [label = < >, shape = none,height=.0,width=.0]
    q1s1,q1s2,q2s1,q2s2,q3s1,q3s2,q4s1 [shape = circle]
    q4s2 [shape = doublecircle]
    0->q1s1
    q1s1->q2s2[label="a,b"]
    q1s2->q2s1[label="a,b"]
    q2s1->q3s2[label="a,b"]
    q2s2->q3s1[label="a,b"]
    q3s1->q4s2[label="a,b"]
    q3s2->q4s1[label="a,b"]
    q4s1->q4s2[label="a,b"]
    q4s2->q4s1[label="a,b"]
    
}
\end{center}

Упростим автомат:

\begin{center}
\digraph[scale=1] {d13} {
    rankdir = LR
    t1,t2,t3 [shape = circle]
    t4 [shape = doublecircle]
    0 [label = < >, shape = none,height=.0,width=.0]
    0->t1
    t1->t2[label="a,b"]
    t2->t3[label="a,b"]
    t3->t4[label="a,b"]
    t4->t3[label="a,b"]
    
}
\end{center}

\begin{center}
$2.3) \quad L_3 = \{w \in \{ a,b \}^* \; | \quad |w|_a \mbox { чётно} \wedge |w|_b \mbox{ кратно трём} \}$ \newline

$L_3_1 = \{w \in \{ a,b \}^* \; | \quad |w|_a \mbox { чётно}\}$ \newline

\digraph[scale = 1] {d14}{
    rankdir=LR
    0 [label = < >, shape = none,height=.0,width=.0]
    q2 [shape = circle]
    q1 [shape = doublecircle]
    0->q1
    q1->q1 [label="b"]
    q1->q2 [label="a"]
    q2->q2 [label="b"]
    q2->q1 [label="a"]
}

$L_3_2 = \{w \in \{ a,b \}^* \; | \quad |w|_a \mbox { чётно}\}$ \newline

\digraph[scale = 1] {d15}{
    0 [label = < >, shape = none,height=.0,width=.0]
    rankdir=LR
    s2,s3 [shape =circle] 
    s1 [shape =doublecircle] 
    0->s1
    s1->s2 [label="b"]
    s1->s1 [label="a"]
    s2->s3 [label="b"]
    s2->s2 [label="a"]
    s3->s1 [label="b"]
    s3->s3 [label="a"]
}

\begin{tabular}{|c|c|c|}
            \hline
            Состояния & a & b \\
            \hline
            \textbf{q1s1} & q2s1 & q1s2\\
            q1s2 & q2s2 & q1s3\\
            q1s3 & q2s3 & q1s1\\
            q2s1 & q1s1 & q2s2\\
            q2s2 & q1s2 & q2s3\\
            q2s3 & q1s3 & q2s1\\
            \hline
\end{tabular}
\newline
\end{center} 

 
Прямое произведение автоматов:

\begin{center}
\digraph[scale=0.7] {d16} {
    rankdir = LR
    0 [label = < >, shape = none,height=.0,width=.0]
    q1s2,q1s3,q2s1,q2s2,q2s3 [shape = circle]
    q1s1 [shape = doublecircle]
    0->q1s1
    q1s1->q2s1[label="a"]
    q1s2->q2s2[label="a"]
    q1s3->q2s3[label="a"]
    q2s1->q1s1[label="a"]
    q2s2->q1s2[label="a"]
    q2s3->q1s3[label="a"]
    
    q1s1->q1s2[label="b"]
    q1s2->q1s3[label="b"]
    q1s3->q1s1[label="b"]
    q2s1->q2s2[label="b"]
    q2s2->q2s3[label="b"]
    q2s3->q2s1[label="b"]
}

$2.4) \quad L_4 = \overline L_3$ \newline

\digraph[scale=0.7] {d17} {
    rankdir = LR
    0 [label = < >, shape = none,height=.0,width=.0]
    q1s2,q1s3,q2s1,q2s2,q2s3 [shape = doublecircle]
    q1s1 [shape = circle]
    0->q1s1
    q1s1->q2s1[label="a"]
    q1s2->q2s2[label="a"]
    q1s3->q2s3[label="a"]
    q2s1->q1s1[label="a"]
    q2s2->q1s2[label="a"]
    q2s3->q1s3[label="a"]
    
    q1s1->q1s2[label="b"]
    q1s2->q1s3[label="b"]
    q1s3->q1s1[label="b"]
    q2s1->q2s2[label="b"]
    q2s2->q2s3[label="b"]
    q2s3->q2s1[label="b"]
}
\newline
\end{center}
Так как $T_4=Q_3\textbackslash T_3 = {q1s2,q1s3,q2s1,q2s2,q2s3}$. \\ 
\begin{center}
$2.5) \quad L_5 = L_2 \textbackslash L_3 = L_2 \textbackslash L_3 = L_2 \cap \overline L_3 = \overline L_3 \times L_2$ 
\newline

\digraph[scale=0.8] {d18} {
    rankdir = LR
    0 [label = < >, shape = none,height=.0,width=.0]
    r2,r3,r4,r5,r6 [shape = doublecircle]
    r1 [shape = circle]
    0->r1
    r1->r4[label="a"]
    r2->r5[label="a"]
    r3->r6[label="a"]
    r4->r1[label="a"]
    r5->r2[label="a"]
    r6->r3[label="a"]
    
    r1->r2[label="b"]
    r2->r3[label="b"]
    r3->r1[label="b"]
    r4->r5[label="b"]
    r5->r6[label="b"]
    r6->r4[label="b"]
}
\digraph[scale=1] {d13} {
    rankdir = LR
    t1,t2,t3 [shape = circle]
    t4 [shape = doublecircle]
    0 [label = < >, shape = none,height=.0,width=.0]
    0->t1
    t1->t2[label="a,b"]
    t2->t3[label="a,b"]
    t3->t4[label="a,b"]
    t4->t3[label="a,b"]
}
\begin{tabular}{ |c|c|c| } 
                \hline
                 Состояния & a (rt) & b (rt) \\
                \hline\hline
                r1t1 & 42 & 22 \\
                \hline 
                r1t2 & 43 & 23 \\
                \hline
                r1t3 & 44 & 24 \\
                \hline
                r1t4 & 43 & 23 \\
                \hline\hline
                r2t1 & 52 & 32 \\
                \hline
                r2t2 & 53 & 33 \\
                \hline
                r2t3 & 54 & 34 \\
                \hline
                r2t4 & 53 & 33 \\
                \hline\hline
                r3t1 & 62 & 12 \\
                \hline
                r3t2 & 63 & 13 \\
                \hline
                r3t3 & 64 & 14 \\
                \hline
                r3t4 & 63 & 13 \\
                \hline
            \end{tabular} \:\:
            \begin{tabular}{ |c|c|c| } 
                \hline
                Состояния & a & b \\
                \hline\hline
                r4t1 & 12 & 52 \\
                \hline
                r4t2 & 13 & 53 \\
                \hline
                r4t3 & 14 & 54 \\
                \hline
                r4t4 & 13 & 53 \\
                \hline\hline
                r5t1 & 22 & 62 \\
                \hline
                r5t2 & 23 & 63 \\
                \hline
                r5t3 & 24 & 64 \\
                \hline
                r5t4 & 23 & 63 \\
                \hline\hline
                r6t1 & 32 & 42 \\
                \hline
                r6t2 & 33 & 43 \\
                \hline
                r6t3 & 34 & 44 \\
                \hline
                r6t4 & 33 & 43 \\
                \hline
            \end{tabular}
          
\digraph[scale=0.5] {d19} {
    0 [label = < >, shape = none,height=.0,width=.0]
    11,12,13,14,21,22,23,31,32,33,41,42,43,51,52,53,54,61,62,63 [shape = circle]
    24,34,44,64 [shape = doublecircle]
    0->11
    11->42[label="a"]
    12,14->43[label="a"]
    13->44[label="a"]
    
    21->52[label="a"]
    22,24->53[label="a"]
    23->54[label="a"]
    
    31->62[label="a"]
    32,34->63[label="a"]
    33->64[label="a"]

    41->12[label="a"]
    42,44->13[label="a"]
    43->14[label="a"]

    51->22[label="a"]
    52,54->23[label="a"]
    53->24[label="a"]

    61->32[label="a"]
    62,64->33[label="a"]
    63->34[label="a"]
    
    11->22[label="b"]
    12,14->23[label="b"]
    13->24[label="b"]
    
    21->32[label="b"]
    22,24->33[label="b"]
    23->34[label="b"]
    
    31->12[label="b"]
    32,34->13[label="b"]
    33->14[label="b"]

    41->52[label="b"]
    42,44->53[label="b"]
    43->54[label="b"]

    51->62[label="b"]
    52,54->63[label="b"]
    53->64[label="b"]

    61->42[label="b"]
    62,64->43[label="b"]
    63->44[label="b"]

}
\end{center}

\section{Задание №3. Построить минимальный ДКА по регулярному выражению}
 
\begin{equation}
(ab + aba)^* a
\end{equation}
 
ДКА для ab:
 
\digraph {Task311}{
    0 [label = < >, shape = none,height=.0,width=.0];
    rankdir=LR;
    node[shape=circle] s0,s1;
    node[shape=doublecircle] s2;
    0->s0;
    s0->s1 [label="a"];
    s1->s2 [label="b"];
}
 
ДКА для aba:
 
\digraph {Task312}{
    rankdir=LR;
    0 [label = < >, shape = none,height=.0,width=.0];
    node[shape=circle] q0,q1,q2;
    node[shape=doublecircle] q3;
    0->q0;
    q0->q1 [label="a"];
    q1->q2 [label="b"];
    q2->q3 [label="a"];
}
 
Автомат объединения aba и ab:
 
\digraph[scale = 0.8] {Task313}{
    rankdir=LR;
    0 [label = < >, shape = none,height=.0,width=.0];
    node[shape=circle] sq1, s0, s1,q0,q1,q2;
    node[shape=doublecircle] sq2;
    0->sq1;
    sq1->s0 [label="eps"]
    sq1->q0 [label="eps"]
    q0->q1 [label="a"];
    q1->q2 [label="b"];
    q2->sq2[label="a"];
    s0->s1 [label="a"];
    s1->sq2 [label="b"];
}
 
НКА с итерациями:
 
\digraph [scale=0.65]{Task314}{
    rankdir=LR;
    0 [label = < >, shape = none,height=.0,width=.0];
    node[shape=circle] sq1,s0,s1,q0,q1,q2;
    node[shape=doublecircle] sq2;
    0->sq1;
    sq1->sq2 [label="eps"];
    sq1->s0 [label="eps"];
    sq1->q0 [label="eps"];
    q0->q1 [label="a"];
    q1->q2 [label="b"];
    q2->sq2 [label="a"];
    s0->s1 [label="a"];
    s1->sq2 [label="b"];
    sq2->sq1[label="eps"];
}
 
НКА с конкатенацией:
 
\digraph [scale=0.65]{Task315}{
    rankdir=LR;
    0 [label = < >, shape = none,height=.0,width=.0];
    node[shape=circle] sq1,s0,s1,q0,q1,q2,sq2;
    node[shape=doublecircle] sq3;
    0->sq1;
    sq1->sq2 [label="eps"];
    sq1->s0 [label="eps"];
    sq1->q0 [label="eps"];
    q0->q1 [label="a"];
    q1->q2 [label="b"];
    q2->sq2 [label="a"];
    s0->s1 [label="a"];
    s1->sq2 [label="b"];
    sq2->sq1[label="eps"];
    sq2->sq3[label="a"];
}

Избавляемся от $\lambda$-переходов:

\digraph [scale=0.65]{Task316}{
    rankdir=LR;
    0 [label = < >, shape = none,height=.0,width=.0];
    node[shape=circle] 1,2,3,4;
    node[shape=doublecircle] 5;
    0->1;
    1->2 [label="a"];
    1->3 [label="a"];
    1->5 [label="a"];
    2->1 [label="b"];
    3->4 [label="b"];
    4->1 [label="a"];
    
}

 \begin{tabular}{|c|c|c|}
            \hline
            Узлы & a & b \\
            \hline
            1 & 235 & -\\
            235 & - & 14\\
            14 & 2351 & -\\
            2351 & 235 & 14\\
            \hline
\end{tabular}


\digraph [scale=0.65]{Task317}{
    rankdir=LR;
    0 [label = < >, shape = none,height=.0,width=.0];
    node[shape=circle] 1,14;
    node[shape=doublecircle] 235,2351;
    0->1;
    1->235 [label="a"];
    235->14 [label="b"];
    14->2351 [label="a"];
    2351->14 [label="b"];
    2351->235 [label="a"];
}

Переименуем вершины для красоты:\\
\digraph [scale=0.7]{Task318}{
    rankdir=LR;
    0 [label = < >, shape = none,height=.0,width=.0];
    node[shape=circle] 1,3;
    node[shape=doublecircle] 2,4;
    0->1;
    1->2 [label="a"];
    2->3 [label="b"];
    3->4 [label="a"];
    4->3 [label="b"];
    4->2 [label="a"];
}

\begin{equation}
a(a(ab)^* b)^* (ab)^*
\end{equation}
 
Построим НКА:
\newline

\digraph[scale=1]{Task321}{
    0 [label = < >, shape = none,height=.0,width=.0];
    node [shape = doublecircle] 2,5;
    node [shape = circle] 1,3,4;
    rankdir=LR; 
    0 -> 1;
    1 -> 2 [label = "a"];
    2 -> 3 [label = "a"];
    2 -> 5 [label = "a"];
    3 -> 4 [label = "a"];
    4 -> 3 [label = "b"];
    3 -> 2 [label = "b"];
    5 -> 2 [label = "b"];
}

Построим эквивалентный ДКА:
\newline

\begin{tabular}{ |c|c|c| } 
    \hline
    Состояния & a & b \\
    \hline\hline
    1 & 2 & - \\
    \hline
    2 & 35 & - \\
    \hline
    35 & 4 & 2 \\
    \hline
    4 & - & 3 \\
    \hline
    3 & 4 & 2 \\
    \hline
\end{tabular}
\newline
\digraph[scale=0.8]{Tasks322}{
    0 [label = < >, shape = none,height=.0,width=.0];
    node [shape = doublecircle]; 2;
    node [shape = circle];
    rankdir=LR; 
    0 -> 1;
    1 -> 2 [label = "a"];
    2 -> 35 [label = "a"];
    35 -> 4 [label = "a"];
    35 -> 2 [label = "b"];
    4 -> 3 [label = "b"];
    3 -> 4 [label = "a"];
    3 -> 2 [label = "b"];
}
 
Минимизируем полученный автомат. Недостижимых вершин нет, а вот вершины 35 и 3 неразличимы, соединим их в вершину 3:\\
\digraph[scale=1]{Task323}{
    0 [label = < >, shape = none,height=.0,width=.0];
    node [shape = doublecircle] 2;
    node [shape = circle];
    rankdir=LR; 
    0 -> 1;
    1 -> 2 [label = "a"];
    2 -> 3 [label = "a"];
    3 -> 4 [label = "a"];
    3 -> 2 [label = "b"];
    4 -> 3 [label = "b"];
} 

Красиво!

\begin{equation}
(a + (a + b)(a + b)b)^*
\end{equation}
\newline
Построим НКА:\\
\digraph[scale=0.8]{Task331}{
    0 [label = < >, shape = none,height=.0,width=.0];
    node [shape = doublecircle] 1;
    node [shape = circle];
    rankdir=LR; 
    0 -> 1;
    1 -> 1 [label = "a"];
    1 -> 2 [label = "a,b"];
    2 -> 3 [label = "a,b"];
    3 -> 1 [label = "b"];
} 
\newline
Построим эквивалентный ДКА:\\


\begin{tabular}{ |c|c|c| } 
    \hline
    Состояния & a & b \\
    \hline\hline
    1 & 12 & 2 \\
    \hline
    12 & 123 & 23 \\
    \hline
    2 & 3 & 3 \\
    \hline
    123 & 123 & 123 \\
    \hline
    23 & 3 & 13 \\
    \hline
    3 & - & 1 \\
    \hline
    13 & 12 & 12 \\
    \hline
\end{tabular}
\newline

\digraph[scale=0.8]{Task332}{
    0 [label = < >, shape = none,height=.0,width=.0];
    node [shape = doublecircle] 1,12,123,13;
    node [shape = circle];
    rankdir=LR; 
    0 -> 1;
    1 -> 12 [label = "a"];
    123 -> 123 [label = "a,b"];
    23 -> 3 [label = "a"];
    1 -> 2 [label = "b"];
    12 -> 23 [label = "b"];
    12 -> 123 [label = "a"];
    2 -> 3 [label = "a,b"];
    23 -> 13 [label = "b"];
    3 -> 1 [label = "b"];
    13 -> 12 [label = "a,b"];
} 
\newline

Разобьем на классы эквивалентности:

\textbf{k0}:\{1,12,123,13\} \{23,2,3\}

\textbf{k1}:\{1,12\} \{123,13\} \{23\} \{3\} \{2\}

\textbf{k2}:\{1\} \{12\} \{123\} \{13\} \{23\} \{3\} \{2\}

Значит наш автомат минимален.

\begin{equation}
(b+c)((ab)^*c + (ba)^*)^*
\end{equation}

Построим ДКА:
        
\digraph[scale=0.6]{Task341}{
    0 [label = < >, shape = none,height=.0,width=.0];
    node [shape = doublecircle] 2 5 7;
    node [shape = circle];
    rankdir=LR; 
    0 -> 1;
    1 -> 2 [label = "b, c"];
    2 -> 3 [label = "a"];
    2 -> 6 [label = "b"];
    3 -> 4 [label = "b"];
    4 -> 5 [label = "c"];
    4 -> 3 [label = "a"];
    5 -> 3 [label = "a"];
    5 -> 6 [label = "b"];
    6 -> 7 [label = "a"];
    7 -> 3 [label = "a"];
    7 -> 6 [label = "b"];
}  

Минимизируем полученный автомат (cостояния 2,5,7 неразличимы - сведем к вершине 257):
     

\digraph[scale=0.6]{Task342}{
    0 [label = < >, shape = none,height=.0,width=.0];
    node [shape = doublecircle] 257;
    node [shape = circle];
    rankdir=LR; 
    0 -> 1;
    1 -> 257 [label = "b, c"];
    257 -> 3 [label = "a"];
    257 -> 6 [label = "b"];
    3 -> 4 [label = "b"];
    4 -> 257 [label = "c"];
    4 -> 3 [label = "a"];
    6 -> 257 [label = "a"];
}  

\begin{equation}
(a+b)^+(aa + abab + bb + baba)(a+b)^+
\end{equation}        

Построим НКА:


\digraph[scale=0.62] {Task351}{
    0 [label = < >, shape = none,height=.0,width=.0];
    node [shape=doublecircle] q11;
    node [shape=circle];
	rankdir=LR;
	0->q1;
    q1->q2 [label="a,b"];
    q2->q1 [label="a,b"];
    q2->q3 [label="a"];
    q2->q4 [label="b"];
    q3->q5 [label="b"];
    q3->q9 [label="a"];
    q4->q6 [label="a"];
    q4->q10 [label="b"];
    q5->q7 [label="a"];
    q6->q8 [label="b"];
    q7->q9 [label="b"];
    q8->q10 [label="a"];
    q9->q11 [label="a,b"];
    q10->q11 [label="a,b"];
    q11->q11 [label="a,b"];
}      

    
Эквивалентный ДКА:

\digraph[scale=0.22] {Task352}{
    0 [label = < >, shape = none,height=.0,width=.0];
    node [shape=doublecircle] q1q3q11, q2q5q11, q1q3q7q11, q2q9q11, q2q5q9q11, q1q4q11, q2q6q11, q1q4q8q11, q2q6q10q11, q2q10q11;
    node [shape=circle];
	rankdir=LR;
	0->q1;
    q1 -> q2 [label="a,b"];
    q2 -> q1q3 [label="a"];
    q2 -> q1q4 [label="b"];
    q1q3 -> q2q9 [label="a"];
    q1q3 -> q2q5 [label="b"];
    q2q5 -> q1q3q7 [label="a"];
    q2q5 -> q1q4 [label="b"];
    q1q4 -> q2q6 [label="a"];
    q1q4 -> q2q10 [label="b"];
    q2q6 -> q1q3 [label="a"];
    q2q6 -> q1q4q8 [label="b"];
    q1q4q8 -> q2q6q10 [label="a"];
    q1q4q8 -> q2q10 [label="b"];
    q2q6q10 -> q1q3q11 [label="a"];
    q2q6q10 -> q1q4q8q11 [label="b"];
    q2q10 -> q1q3q11 [label="a"];
    q2q10 -> q1q4q11 [label="b"];
    q1q3q7 -> q2q9 [label="a"];
    q1q3q7 -> q2q5q9 [label="b"];
    q2q9 -> q1q3q11 [label="a"];
    q2q9 -> q1q4q11 [label="b"];
    q1q3q11 -> q2q9q11 [label="a"];
    q1q3q11 -> q2q5q11 [label="b"];
    q2q5q11 -> q1q3q7q11 [label="a"];
    q2q5q11 -> q1q4q11 [label="b"];
    q2q5q9 -> q1q4q11 [label="b"];
    q2q5q9 -> q1q3q7q11 [label="a"];
    q2q5q9 -> q1q4q11 [label="b"];
    q1q3q7q11 -> q2q9q11 [label="a"];
    q1q3q7q11 -> q2q5q9q11 [label="b"];
    q2q9q11 -> q1q4q11 [label="b"];
    q2q5q9q11 -> q1q3q7q11 [label="a"];
    q2q5q9q11 -> q1q4q11 [label="b"];
    q1q4q11 -> q2q6q11 [label="a"];
    q1q4q11 -> q2q10q11 [label="b"];
    q2q6q11 -> q1q4q8q11 [label="b"];
    q1q4q8q11 -> q2q6q10q11 [label="a"];
    q1q4q8q11 -> q2q10q11 [label="b"];
    q2q6q10q11 -> q1q3q11 [label="a"];
    q2q6q10q11 -> q1q4q8q11 [label="b"];
    q2q10q11 -> q1q3q11 [label="a"];
    q2q10q11 -> q1q4q11 [label="b"];
}       


Минимизируем ДКА:

\digraph[scale=0.56] {task3ans5}{
    0 [label = < >, shape = none,height=.0,width=.0];
    node [shape=doublecircle] h9;
    node [shape=circle];
	rankdir=LR;
	0->h1;
	h1 -> h2 [label="a, b"]
    h2 -> h3 [label="a"]
    h2 -> h4 [label="b"]
    h3 -> h8 [label="a"]
    h3 -> h5 [label="b"]
    h4 -> h6 [label="a"]
    h4 -> h8 [label="b"]
    h5 -> h7 [label="a"]
    h5 -> h4 [label="b"]
    h6 -> h3 [label="a"]
    h6 -> h7 [label="b"]
    h7 -> h8 [label="a,b"]
    h8 -> h9 [label="a,b"]
    h9 -> h9 [label="a,b"]
}       

    
Автомат минимален.


\section{Задание №4. Определить, является ли следующие языки регулярными или нет:}
    

\begin{enumerate}
    \item \(L=\{(aab)^{n}b(aba)^{m} : n \geqslant 0, m \geqslant 0\}\)
    
Язык регулярный, т.к. по нему можно построить ДКА:

\digraph[scale=0.5]{g41}{
    0 [label = < >, shape = none,height=.0,width=.0];
    node [shape = doublecircle] 5 8;
    node [shape = circle];
    rankdir=LR; 
    0 -> 1;
    1 -> 2 [label = "a"];
    1 -> 5 [label = "b"];
    2 -> 3 [label = "a"];
    3 -> 4 [label = "b"];
    4 -> 5 [label = "b"];
    4 -> 2 [label = "a"];
    5 -> 6 [label = "a"]; 
    6 -> 7 [label = "b"];
    7 -> 8 [label = "a"];
    8 -> 5 [label = "a"];
}  



    \item \(L = \{uaav : u \in \{a, b\}^*, \; v \in \{a, b\}^*, |u|_b \geqslant |v|_a\}\)
    
    Фиксируем \(\forall n \in \mathbb{N} \) и рассматриваем слово \(\omega = b^{n}aaa^{n}, \; |\omega| = 2n + 2 \geq n\). Рассмотрим все разбиения этого слова \(\omega = xyz\) такие, что \(|y| \neq 0, \; |xy| \leq n\):
    $$x = b^{k}, \; y = b^{l}, \; z = b^{n - k - l}aaa^n,$$ 
    
    \begin{center}
        где \(1 \leq k + l \leq n \; \wedge \; l > 0\)
    \end{center} 
    
    Других разбиений нет.
    Для любого из таких разбиений слово \(xy^0z \notin L\). Лемма о разрастании не выполняется, значит \(L\) нерегулярный язык.
    
    \item \(L = \{a^mw : w \in \{a, b\}^{*}, \; 1 \geqslant |w|_b \geqslant m\}\)
    
    
    Фиксируем \(\forall n \in \mathbb{N} \) и рассматриваем слово \(\omega = a^nb^n, \; |\omega| = 2n \geqslant n\). Рассмотрим все разбиения этого слова \(\omega = xyz\) такие, что \(|y| \neq 0, \; |xy| \leq n\):
    $$x = a^{l}, \; y = a^{m}, \; z = a^{n-l-m}b^{n},$$ 
    
    \begin{center}
        где \(l + k \leqslant n \; \wedge \; m \ne 0\)
    \end{center} 
    
    Других разбиений нет. Выполним накачку: 
    $$xy^{i}z = a^{l}(a^{m})^{i}a^{n-l-m}b^{n} = a^{n-mi}b^{n} \notin L, \; i 
    \geqslant 0 \in \mathbb{N} $$
    Лемма о разрастании не выполняется, значит \(L\) нерегулярный язык.
    
    \item \(L = \{a^{k}b^{m}a^{n} : k = n \vee m > 0\}\)
    
    
   Фиксируем \(\forall n \in \mathbb{N} \) и рассматриваем слово \(\omega = a^nba^n, \; |\omega| = 2n + 1 \geqslant n\). Рассмотрим все разбиения этого слова \(\omega = xyz\) такие, что \(|y| \neq 0, \; |xy| \leq n\):
    $$x = a^{k}, \; y = a^{m}, \; z = a^{n-k-m}ba^{n},$$ 
    
    \begin{center}
        где \(k + m \leqslant n \; \wedge \; m \ne 0\)
    \end{center} 
    
    Дргуих разбиений нет.
    Выполняем накачку: 
    $$xy^{i}z = a^{k}(a^{m})^{i}a^{n-k-m}ba^{n} = a^{n+m(i-1)}ba^{n} \notin L, \; i 
    \geqslant 2 \in \mathbb{N} $$
    Получили противоречие, лемма о разрастании не выполняется, значит \(L\) нерегулярный язык.
    
    \item \(L = \{ucv : u \in \{a, b\}^*, \; v \in \{a, b\}^*, u \ne v^R \}\)
    
    
    Фиксируем \(\forall n \in \mathbb{N} \) и рассматриваем слово \(\omega = (ab)^nc(ab)^n = \alpha_1\alpha_2...\alpha_{4n+1}, \; |\omega| = 4n + 1 \geqslant n\). Рассмотрим все разбиения этого слова \(\omega = xyz\) такие, что \(|y| \neq 0, \; |xy| \leq n\):
    $$x = \alpha_1\alpha_2...\alpha_k, \; y = \alpha_{k+1}...\alpha_{k+m}, \; z = \alpha_{k+m+1}...\alpha_{4n+1}c(ab)^n,$$
    
    \begin{center}
        где \(k + m \leqslant n \; \wedge \; m \ne 0\)
    \end{center} 
    
    Других разбиений нет. Выполняем накачку: 
    $$xy^{i}z = (\alpha_1\alpha_2...\alpha_k)(\alpha_{k+1}...\alpha_{k+m})^i(\alpha_{k+m+1}...\alpha_{4n+1}c(ab)^n)$$
    При \(i = 2\) имеем:
    $$xy^{2}z = (\alpha_1\alpha_2...\alpha_k)(\alpha_{k+1}...\alpha_{k+m})^2(\alpha_{k+m+1}...\alpha_{4n+1}c(ab)^n) \notin L$$
    Лемма о разрастании не выполняется, значит \(L\) нерегулярный язык.

\end{enumerate}


\end{document}
